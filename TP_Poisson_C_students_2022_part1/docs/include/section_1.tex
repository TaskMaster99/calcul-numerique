\section{Methode directe et stockage bande}
\subsection{Exercice 3}
\begin{enumerate}
    \item Une matrice doit être en 1D (matrice linéarisée).
    \item LAPACK\_COL\_MAJOR spécifie le type de rangement des données, i.e. colonne.
    \item Le \textit{leading dimension} correspond à l'offset pour passer de ligne en ligne ou de colonne en colonne.\\
          Pour un type de rangement en column major, le leading dimension est le nombre de lignes.\\
          Pour un type de rangement en row major, le leading dimension est le nombre de colonnes.
    \item dgbmv est une fonction qui effectue une multiplication matrice vecteur
    \item dgbtrf est une fonction qui effectue une factorisation LU.
    \item dgbtrs est une focntion qui effectue une resolution d'un systeme lineaire Ax = b
    \item dgbsv est une fonction qui effectue une resolution d'un systeme lineaire Ax = b avec A une matrice bande
\end{enumerate}